The problem of efficiently checking the correctness of a computation performed by an untrusted party has been central in Cryptography and Complexity Theory for the last 30 years since the introduction of Interactive Proofs by Babai and Goldwasser, Micali and Rackoff \cite{babai,gmr}. 

The rise of the {\em cloud computing} paradigm has refocused our community's research effort towards finding solutions that are efficient and feasible in practice. How do we check the integrity of data that is stored remotely? How do we check computations performed on this remotely stored data? How can a client do this in the most efficient way possible?

Under the generic name of {\sf Verifiable Outsourced Computation} we have a very active research area in Cryptography and Network Security (see \cite{wb15} for a survey), to design protocols where it is impossible (under suitable cryptographic assumptions) for a provider to "cheat" in the above scenarios. While much progress has been done in this area, we are still far from solutions that can be deployed in practice. 

Part of the reason is that Cryptographers consider a very strong adversarial model that prevents {\sf any} adversary from cheating. A different approach is to restrict ourselves to {\em rational adversaries}, whose motivation is not just to disrupt the protocol or computation, but simply to maximize a well defined utility function (e.g. profit).

\medskip
\noindent
{\sc Our Contribution:}
This paper presents new protocols for the verification of {\em non-deterministic space-bounded polytime computations} against a rational adversary. More specifically consider a language $L \in \NTISP(T(n), S(n))$, i.e. recognized by a non-deterministic Turing Machine $M_L$ which runs in time $T(n)$ and space $S(n)$. 
We construct a protocol where a rational prover can
convince the verifier that $x \in L$ with the following properties: 
\begin{itemize}
\item The verifier runs in time $O(S(n) \log n)$
\item The protocol has $O(\log n)$ rounds and communication complexity $O(S(n) \log n)$
\item The prover simply runs $M_L(x)$ and stores all the intermediate configurations (i.e. requires space $O(S(n) T(n))$
\end{itemize}
Note that for polytime computations in sublinear space this gives a protocol in which the verifier is sublinear-time. 

\subsection{Relationship to Previous Work}
\label{sec:prior}

\noindent{{\sc Rational Proofs.}
The notion of proof that we use in our work is the concept of {\sf Rational Proofs} introduced by Azar and Micali in \cite{am} and refined in subsequent papers \cite{am1,ratargs}. 

In a Rational Proof, given a function $f$ and an input $x$, the server returns the value $y=f(x)$, and (possibly) some auxiliary information, to the client. The client will in turn 
pay the server for its work with a reward which is a function of the messages 
sent by the server and some randomness chosen by the client.  The crucial 
property is that this reward is maximized in expectation when the server 
returns the correct value $y$. Clearly a rational prover who is only interested 
in maximizing his reward, will always answer correctly. 

The most striking feature of Rational Proofs is their simplicity. For example in \cite{am}, Azar and Micali show {\sf single-message} Rational Proofs for any problem in $\#P$, where an (exponential-time) prover convinces a (poly-time) verifier of the number of satisfying assignment of a Boolean formula. 

For the case of "real-life" computations, Azar and Micali in \cite{am1} present $d$-round Rational Proofs for functions computed by (uniform) Boolean circuits of depth $d$, for $d=O(\log n)$. In this case the Verifier runs in logarithmic time.

Recent work \cite{ratsumchecks} shows how to obtain Rational Proofs with sublinear verifiers for languages in NC. Recalling that $\L \subseteq \NL \subseteq \NC_2$, one can use the protocol  in \cite{ratsumchecks} to verify a logspace polytime computation (deterministic or nondeterministic) in $O(\log^2 n )$ rounds and $O(\log^2 n )$ verification. For these computations our results compare
favorably to the one in \cite{ratsumchecks} in at least one aspect: our protocol requires $O(\log n )$ rounds and has the same verification complexity.

% Comparison between classes
This paper extends the line of work on rational proofs for "real-life" computation classes. We introduce the first rational proof for $\SC$ (also known as $\DTISP(T(n), S(n))$) with polylogarithmic verification and logarithmic rounds. To compare this with the results in \cite{ratsumchecks}, notice it is believed that $\NC \not = \SC$ and that the two classes are actually incomparable (see \cite{SCcompleteness} for a discussion).
Finally, our results provide the first efficient rational proof for the non-deterministic class $\NSC = \NTISP(\poly(n), \polylog(n) )$ . 

\medskip
\noindent{\sc Interactive Proofs.}
Obviously a "traditional" interactive proof (where security holds against any adversary, even a computationally unbounded one) would work in our model. In this case the most relevant result is 
the recent independent work in \cite{rrr16} that presents breakthrough protocols for the deterministic restriction of the class of language we consider. If $L$ is a language which is recognized by a deterministic Turing Machine $M_L$ which runs in time $T(n)$ and space $S(n)$, then their protocol has the following properties: 
\begin{itemize}
\item The verifier runs in 
$O(\poly(S(n)) + n \cdot\polylog(n))$ time;
\item The prover runs in polynomial time;
\item The protocol runs in {\em constant} rounds, with communication complexity $O({\sf poly}(S(n)n^{\delta})$ for a constant $\delta$.
\end{itemize}
Apart from round complexity (which is the impressive breakthrough of the result in \cite{rrr16}) our protocols fares better in all other categories. Note in particular that a sublinear space computation does not necessarily yield a sublinear-time verifier in 
\cite{rrr16}. On the other hand, we stress that our protocol only considers weaker rational adversaries. 

\medskip
\noindent{\sc Other Related Work.}
There is a large class of protocols for {\em arguments} of correctness (e.g. \cite{ggp10,ggpr13,krr14}) even in the rational model \cite{ratargs,ratsumchecks}. Recall that in an argument, security is achieved only against computationally bounded prover. In this case we can even achieve non-interactive solutions. But, in this paper, we are focused on what we can achieve without using computational assumptions.

% Halpern and Pass
Other works in theoretical computer science have studied the connections between cost of computation and utility in decision problems.
The work in \cite{halpern2011don} proposes a framework for \emph{computational decision problems}, where the Decision Maker's (DM) utility depends on the algorithm chosen for computing its strategy.
The Decision Maker runs the algorithm, assumed to be a Turing Machine, on the input to the computational decision problem.
The output of the algorithm determines the DM's strategy. 
Thus the choice of the DM reduces to the choice of a Turing Machine from a certain space. The DM will have beliefs on the running time (cost) of each Turing machine. The actual cost of running the chosen TM will affect the DM's reward.
Rational proofs with costly computation could be formalized in the language of \emph{computational decision problems} in \cite{halpern2011don}. There are similarities between the approach in this
work and that in \cite{halpern2011don}, as both take into account the cost of computation in a decision problem.

\begin{comment}
The cost models presented in our work could be extended by considering different type of cost functions, such as the ones in \cite{halpern2011don}. For example, we assume a cost function that is linear in the number of steps (transitions) carried out. A different approach could consider non-linear cost functions. What assumptions would be required to prove sequential composability when the cost of the computation increases, say, quadratically in the number of steps? 
And what are plausible cost functions in the real world?
\end{comment}




\subsection{Open Problems and Future Research}


The problem of interactive proofs for any polynomial-time computable function with an efficient verifier (e.g. quasi-linear) remains still open. It would be nice to be able to construct such proofs even simply in the rational model. 

\medskip
\noindent
In \cite{cg15} we presented a critique of the rational proof model in the case of "repeated executions with a budget". This model arises in the context of "volunteer computations" (\cite{seti,folding}) where many computational tasks are outsourced and provers compete in solving as many as possible to obtain rewards ). In this scenario assume that a prover has a certain budget $B$ of "computational effort": how can we guarantee that the rational strategy is to provide the correct answer in {\em all} the proof he provides? The notion of rational proof guarantees that if the prover engages in a single rational proof then it is in his best interest to provide the correct output. But in \cite{cg15} we showed that in the presence of many computations, it might be more profitable for the prover to use his budget $B$ to provide many incorrect answers than to provide a single correct answer. That's because incorrect (e.g. random) answers are "cheaper" to compute than the correct one and with the same budget $B$ the prover can provide many of them while the entire budget might be necessary to solve a single problem correctly. If the difference in reward between correct and incorrect answers is not high enough then many incorrect answers may be more profitable and a rational prover will choose that strategy. 

% XXX: Argue more about this?
The problem seems to be inherent in protocols based on scoring rules such as Brier's rule \cite{brier} (many, though not all, of the protocols in 
\cite{am,am1,ratargs,ratsumchecks} are based on such scoring rules). 

The protocol we present here can be thought of as an extension of the protocol in \cite{cg15}, which is based on a different approach. The Verifier does a limited check on the computation provided by the prover. This check is designed to be efficient (to keep the Verifier simple) and to catch a cheating prover with probability at least $1/{\sf poly}$. If the prover is not caught it gets a fixed reward $R$. This will guarantee that in the single-execution case the rational prover will always choose to provide correct answers and collect $R$ (since the cheating prover has expected reward at most $(1-1/{\sf poly})R$). 



% Interactive proofs

% XXX: From older version (TCC)
As we discussed in \cite{cg15} this type of protocols has a better chance to be secure (against rational adversaries) even in the "repeated" case, provided certain conditions happen. An interesting line of research which we plan to explore is to determine under what circumstances our new protocol for space bounded computation can be proven secure in the repeated case. 
