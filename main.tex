
\documentclass{llncs}
\pagestyle{plain}
% XXX: Page numbering is not working with this either

\usepackage{times}
\usepackage{complexity}
\usepackage{url}
\usepackage{latexsym}
\usepackage[utf8]{inputenc}
\usepackage{framed}
\usepackage{natbib}
\usepackage{graphicx}
%\usepackage{amsthm}
\usepackage{amssymb}
\usepackage{amsmath}
\usepackage{verbatim}
\usepackage{enumitem}
\usepackage{verbatim}
%\usepackage{accents}
\usepackage{bm}
%\usepackage[normalem]{ulem}
%\setlength\titlebox{6.5cm}    % You can expand the title box if you
% really have to

\DeclareMathOperator*{\argmin}{arg\,min}
\DeclareMathOperator*{\argmax}{arg\,max}

\title{Sequentially Composable Rational Proofs for Space Bounded Computations}

\author{}

\institute{}
\date{}

\begin{document}
\maketitle
\input{mycommands.tex}

%% Shortcuts specific to Composition
\newcommand{\protOne}{\pi^{L_2}_{1}}
\newcommand{\protTwo}{\pi_{2}}
\newcommand{\rewGap}{\Delta}
\newcommand{\STEP}{\function{STEP}}

\newcommand{\disTransc}{\tilde{\Tau}}


% Question: Come rendere l'abstract (e il paper) il piu' Eurocrypt-appetibile possibile?
% XXX: The abstract should be rewritten
\begin{abstract}
We present new protocols for the verification of {\em non-deterministic space bounded polytime computations} against a rational adversary.
For non-deterministic polytime computations in sublinear space our protocol requires only a verifier running in sublinear-time. Our protocols can be proven 
secure even when composed sequentially, a scenario where many previously proposed rational proofs fail. Our results are the first rational proofs not based on the circuit model of computation, and the first sequentially composable protocols for a well-defined language class. 
%Unlike previous rational protocols in literature, our protocol can be used for repeated delegations when the computation is costly for the prover. % XXX: %Citations to be included?
\end{abstract}


% -- BEGIN Old abstract --
\begin{comment}
\begin{abstract}
This paper presents new protocols for the verification of {\em non-deterministic space-bounded polytime computations} against a rational adversary. More specifically consider a language $L \in \NTISP(T(n), S(n))$, i.e. a language recognized by a non-deterministic Turing Machine $M_L$ which runs in time $T(n)$ and space $S(n)$. 
We construct a protocol where a rational prover can
convince the verifier that $x \in L$ with the following properties: 
\begin{itemize}
\item The verifier runs in time $O(S(n) \log n)$;
\item The protocol terminates in $O(\log n)$ rounds and communication complexity $O(S(n) \log n)$;
\item The prover simply runs $M_L(x)$ and stores all the intermediate configurations (i.e. requires space $O(S(n) T(n))$.
\end{itemize}
Note that for polytime computations in sublinear space this gives a protocol in which the verifier is sublinear-time. 
\end{abstract}
\end{comment}
% -- END Old abstract --

\section{Introduction}
The problem of efficiently checking the correctness of a computation performed by an untrusted party has been central in Cryptography and Complexity Theory for the last 30 years since the introduction of Interactive Proofs by Babai and Goldwasser, Micali and Rackoff \cite{babai,gmr}. 

The rise of the {\em cloud computing} paradigm has refocused our community's research effort towards finding solutions that are efficient and feasible in practice. How do we check the integrity of data that is stored remotely? How do we check computations performed on this remotely stored data? How can a client do this in the most efficient way possible?

{\sf Verifiable Outsourced Computation} is now a very active research area in Cryptography and Network Security (see \cite{wb15} for a survey), to design protocols where it is impossible (under suitable cryptographic assumptions) for a provider to "cheat" in the above scenarios. While much progress has been done in this area, we are still far from solutions that can be deployed in practice. 

Part of the reason is that Cryptographers consider a very strong adversarial model that prevents {\sf any} adversary from cheating. A different approach is to restrict ourselves to {\em rational adversaries}, whose motivation is not just to disrupt the protocol or computation, but simply to maximize a well defined utility function (e.g. profit).

\subsection{Rational Proofs}

In our work we use the concept of {\sf Rational Proofs} introduced by Azar and Micali in \cite{am} and refined in subsequent papers \cite{am1,ratargs}. 

In a Rational Proof, given a function $f$ and an input $x$, the server returns the value $y=f(x)$, and (possibly) some auxiliary information, to the client. The client will in turn 
pay the server for its work with a reward which is a function of the messages 
sent by the server and some randomness chosen by the client.  The crucial 
property is that this reward is maximized in expectation when the server 
returns the correct value $y$. Clearly a rational prover who is only interested 
in maximizing his reward, will always answer correctly. 

The most striking feature of Rational Proofs is their simplicity. For example in \cite{am}, Azar and Micali show {\sf single-message} Rational Proofs for any problem in $\#P$, where an (exponential-time) prover convinces a (poly-time) verifier of the number of satisfying assignment of a Boolean formula. 

For the case of "real-life" computations, Azar and Micali in \cite{am1} consider the case of efficient provers (i.e. poly-time) and "super-efficient" (log-time) verifiers and present $d$-round Rational Proofs for functions computed by (uniform) Boolean circuits of depth $d$, for $d=O(\log n)$. 
% In this case the Verifier runs in logarithmic time.

Recent work \cite{ratsumchecks} shows how to obtain Rational Proofs with sublinear verifiers for languages in NC. Recalling that $\L \subseteq \NL \subseteq \NC_2$, one can use the protocol  in \cite{ratsumchecks} to verify a logspace polytime computation (deterministic or nondeterministic) in $O(\log^2 n )$ rounds and $O(\log^2 n )$ verification.


\medskip
\noindent
{\sc Compositions of Rational Proofs.}
In \cite{cg15} the authors present a critique of the rational proof model in the case of "repeated executions with a budget". This model arises in the context of "volunteer computations" (\cite{seti,folding}) where many computational tasks are outsourced and provers compete in solving as many as possible to obtain rewards. In this scenario assume that a prover has a certain budget $B$ of "computational effort": how can one  guarantee that the rational strategy is to provide the correct answer in {\em all} the proof he provides? The notion of rational proof guarantees that if the prover engages in a single rational proof then it is in his best interest to provide the correct output. But in \cite{cg15} the authors show that in the presence of many computations, it might be more profitable for the prover to use his budget $B$ to provide many incorrect answers than to provide a single correct answer. That's because incorrect (e.g. random) answers are "cheaper" to compute than the correct one and with the same budget $B$ the prover can provide many of them while the entire budget might be necessary to solve a single problem correctly. If the difference in reward between correct and incorrect answers is not high enough then many incorrect answers may be more profitable and a rational prover will choose that strategy, and indeed this is the case for many of the protocols in \cite{am,am1,ratargs,ratsumchecks}. 

A stronger notion of {\em sequentially composable rational proofs} is put forward in \cite{cg15} which avoids the above problem and guarantees that the rational strategy is always the one to provide correct answers. Protocols satisfying this stronger notion are also presented for a subset of bounded-depth circuit computations. 

\subsection{Our Contribution}

This paper presents new protocols for the verification of {\em non-deterministic space-bounded polytime computations} against a rational adversary. More specifically consider a language $L \in \NTISP(T(n), S(n))$, i.e. recognized by a non-deterministic Turing Machine $M_L$ which runs in time $T(n)$ and space $S(n)$. 
We construct a protocol where a rational prover can
convince the verifier that $x \in L$ with the following properties: 
\begin{itemize}
\item The verifier runs in time $O(S(n) \log n)$
\item The protocol has $O(\log n)$ rounds and communication complexity $O(S(n) \log n)$
\item The prover simply runs $M_L(x)$ and stores all the intermediate configurations (i.e. requires space $O(S(n) T(n))$
\end{itemize}
Our protocol can be proven secure in {\bf both} the stand-alone model of \cite{am} and the sequentially composable definition of \cite{cg15}. 

For the case of "real-life" computations (i.e. poly-time prover and poly-log verifier) we 
note that for computations in sublinear space our general results yields a protocol in which the verifier is sublinear-time. More specifically, we introduce the first rational proof for $\SC$ (also known as $\DTISP(\poly(n), \polylog(n))$) with polylogarithmic verification and logarithmic rounds. Moreover, our results provide the first efficient rational proof for the non-deterministic class $\NSC = \NTISP(\poly(n), \polylog(n) )$ . 

To compare this with the results in \cite{ratsumchecks}, we note that it is believed that $\NC \not = \SC$ and that the two classes are actually incomparable (see \cite{SCcompleteness} for a discussion).
 For these computations our results compare
favorably to the one in \cite{ratsumchecks} in at least one aspect: our protocol requires $O(\log n )$ rounds and has the same verification complexity.

\subsection{The Landscape of Rational Proof Systems}

Rational Proof systems can be divided in roughly two categories, both of them presented in the original work \cite{am}. 

\medskip
\noindent
{\sc Scoring Rules.}
The more "novel" approach in \cite{am} uses {\em scoring rules} to compute the reward paid by the verifier to the prover. A scoring rule is used to asses the "quality" of a prediction of a randomized process. Assume that the prover declares that a certain random variable $X$ follows a particular probability distribution $D$. The verifier runs an "experiment" (i.e. samples the random variable in question) and computes a "reward" based on the distribution $D$ announced by the prover and the result of the experiment. A scoring rule is maximized if the prover announced the real distribution followed by $X$. The novel aspect of many of the protocols in \cite{am} was how to cast the computation of $y=f(x)$ as the announcement of a certain distribution $D$ that could be tested efficiently by the verifier and rewarded by a scoring rule. 

A simple example is the protocol for $\#P$ in \cite{am} (or its "scaled-down" version for threshold gates). Given a Boolean formula $\Phi(x_1,\ldots,x_n)$ the prover announces the number $m$ of satisfying assignments. This can be interpreted as the prover announcing that if one chooses an assignment at random it will be a satisfying one with probability $m \cdot 2^{-n}$. The verifier then chooses a random assignment and checks if it satisfies $\Phi$ or not and uses $m$ and the result of the test to compute the reward via a scoring rule. Since the scoring rule is maximized by the announcement of the correct $m$, a rational prover will announce the correct value. 

As pointed out in \cite{cg15} the problem with the scoring rule approach is that the reward declines slowly as the distribution announced by the Prover becomes more and more distant from the real one. The consequence is that incorrect results still get a substantial reward, even if not a maximal one. Since those incorrect results can be computed faster than the correct one, a Prover with "budget" $B$ might be incentivized to produce many incorrect answers instead of a single correct one. All of the scoring rule based protocols in \cite{am,am1,ratargs,ratsumchecks} suffer from this problem. 

\medskip
\noindent
{\sc Weak Interactive Proofs.}
Rational proofs can also be achieved via "weak" interactive proofs, where the prover is caught cheating with non-negligible probability (this is basically the {\em covert adversary} model for multiparty computation introduced in \cite{AL10}.) Indeed for such proofs we can always pay a fixed reward $R$ to the prover unless we catch him cheating in which case we pay $0$. These are rational proofs since obviously the expected reward of the prover is maximized by the honest behavior. Some of the proofs in \cite{am} and the proofs in \cite{cg15} are weak interactive proofs. Those proofs also turn out to be secure in the sequential model of \cite{cg15} (under appropriate assumptions). The protocol in this work is also a weak interactive proof. 

\medskip
\noindent
{\sc Comparison.}
When it comes to sequentially composability the work in \cite{cg15} and this work seem to point out that scoring rule based protocols are inherently insecure and cannot be used. 

But even if we consider the "stand-alone" model it is not clear which approach is more powerful. For every language class that admits a scoring rule based protocol we also have a weak interactive proof with similar performance metrics (i.e. number of rounds, verifier efficiency, etc.). Our result in this paper present the first example of a language class for which we have rational proofs based on weak interactive proofs but no example of a scoring rule based protocol exist. So our results seem to suggest that the weak interactive proof approach might be the more powerful technique. 

\subsection{Other Related  Work}
\label{sec:prior}

{\sc Interactive Proofs.}
Obviously a "traditional" interactive proof (where security holds against any adversary, even a computationally unbounded one) would work in our model. In this case the most relevant result is 
the recent independent work in \cite{rrr16} that presents breakthrough protocols for the deterministic restriction of the class of language we consider. If $L$ is a language which is recognized by a deterministic Turing Machine $M_L$ which runs in time $T(n)$ and space $S(n)$, then their protocol has the following properties: 
\begin{itemize}
\item The verifier runs in 
$O(\poly(S(n)) + n \cdot\polylog(n))$ time;
\item The prover runs in polynomial time;
\item The protocol runs in {\em constant} rounds, with communication complexity $O({\sf poly}(S(n)n^{\delta})$ for a constant $\delta$.
\end{itemize}
Apart from round complexity (which is the impressive breakthrough of the result in \cite{rrr16}) our protocols fares better in all other categories. Note in particular that a sublinear space computation does not necessarily yield a sublinear-time verifier in 
\cite{rrr16}. On the other hand, we stress that our protocol only considers weaker rational adversaries. 

\medskip
\noindent{\sc Computational Arguments.}
There is a large class of protocols for {\em arguments} of correctness (e.g. \cite{ggp10,ggpr13,krr14}) even in the rational model \cite{ratargs,ratsumchecks}. Recall that in an argument, security is achieved only against computationally bounded prover. In this case we can even achieve non-interactive solutions. But, in this paper, we are focused on what we can achieve without using computational assumptions.

\medskip
\noindent
{\sc Computational Decision Theory.}
Other works in theoretical computer science have studied the connections between cost of computation and utility in decision problems.
The work in \cite{halpern2011don} proposes a framework for \emph{computational decision problems}, where the Decision Maker's (DM) utility depends on the algorithm chosen for computing its strategy.
The Decision Maker runs the algorithm, assumed to be a Turing Machine, on the input to the computational decision problem.
The output of the algorithm determines the DM's strategy. 
Thus the choice of the DM reduces to the choice of a Turing Machine from a certain space. The DM will have beliefs on the running time (cost) of each Turing machine. The actual cost of running the chosen TM will affect the DM's reward.
Rational proofs with costly computation could be formalized in the language of \emph{computational decision problems} in \cite{halpern2011don}. There are similarities between the approach in this
work and that in \cite{halpern2011don}, as both take into account the cost of computation in a decision problem.







\section{Preliminaries}
\subsection{Rational Proofs}

% XXX: Do we need 
In the following we will adopt a "concrete-security" version of the "asymptotic" definitions and theorems in \cite{am1,ratargs}. We assume the reader is familiar with the notion of interactive proofs \cite{gmr}. 

\noindent
\begin{definition}[Rational Proof]
\label{def:RP-delta}
\label{def:RP}
A function $f:$ $\bit^n$ $\to$ $\bit^n$ admits a rational proof if there exists an interactive proof $(P,V)$ and a randomized reward function
$\rew : \bits \to \posreals$ such that

\begin{enumerate}
\item For any input $x \in 
\bit^n$, $\Pr[\out((P,V)(x)) = f(x)] = 1.$

\item For every prover $\disP$, and for any input $x \in 
\bit^n$ there exists a $\delta_{\disP}(x) \geq 0$ such that 
$ \expRewProtDis + \delta_{\disP}(x) \leq \expRewProtHon. $
\end{enumerate}
The expectations and the probabilities are taken over the random coins of the prover and verifier.
\end{definition} 


\medskip
\noindent
Let $\epsilon_{\disP} = \Pr[\out((P,V)(x)) \neq f(x)]$. 
Following \cite{ratargs} we define the {\sf reward gap} as 
\[ \Delta(x) = min_{P^* : \epsilon_{P^*}=1}[\delta_{P^*}(x)]  \]
i.e. the minimum reward gap over the provers that always report the incorrect value. 
It is easy to see that for arbitrary prover $\disP$ we have $\delta_{\disP}(x) \geq 
\epsilon_{\disP} \cdot \Delta(x)$. Therefore it suffices to prove that a protocol has 
a strictly positive reward gap $\Delta(x)$ for all $x$. 


\begin{myrem}
\label{rem:asy}
{\em If we are interested in an asymptotic treatment, it is important to notice that as long as $\Delta(x) \geq 1/{\sf poly}(|x|)$ then it is possible to keep a polynomial reward budget, and maximize the honest prover profit against all provers who cheat with a substantial probability $\epsilon_{\disP} \geq 1/{\sf poly'}(|x|)$.}
\end{myrem}

\begin{definition}[\cite{am1,ratargs}]
The class $\DRMA[r, c, T]$ (Decisional Rational Merlin Arthur)
is the class of boolean functions $f : \bits \to \bit$ admitting a rational proof $\Pi = (P,V, \rew)$ s.t. on input $x$:
\begin{itemize}
    \item $\Pi$ terminates in $r(|x|)$ rounds;
    \item The communication complexity of $P$ is $c(|x|)$;
    \item The running time of $V$ is $T(|x|)$;
    \item $\Pi$ has noticeable reward gap.
\end{itemize}
\end{definition}


\begin{comment}
A composition theorem (in its weak version: for yes/no Rational Proofs):
\begin{theorem}
Let $\protOne = (P_1, V_1^{L_2})$ be a (yes/no) rational proof for language $L_1$ with noticeable reward gap and let $V^{L_2}$ have oracle access to language $L_2$ with at most $O(1)$ queries.
Let $\protTwo (P_2, V_2)$ be a (yes/no) rational proof for $L_2$ with noticeable reward gap.
Then there exists a (yes/no) rational proof $\pi$ for $L_1$ with noticeable reward gap.
Moreover if $\protOne$ and $\protTwo$ have round, communication and verification complexity respectively $r_1(n), cc_1(n), T_1(n)$  and $r_2(n), cc_2(n), T_2(n)$ then language $L_1 \in DRMA[r_1(n) + O(r_2(n)), cc_1(n) + O(cc_2(n)), T_1(n) + O(T_2(n))]$
\end{theorem}
\end{comment}

% TODO: Observe that since rational proofs with poly budget without noticeable rew gap "hardly" make sense, we will implicitly consider rational proofs, always rational proofs with noticeable rew. gap.



\subsection{Sequential Composability}

% NOTE: Mostly all from last published paper

Intuitively sequential composability attempts to capture the following intuition: the reward of the honest prover $P$ must always be larger than the total  reward of any prover $\disP$ that invests less computation cost than $P$.


It is not trivial to give a definition of sequential composability, since it is not possible to claim the above for {\em any} prover $\disP$ and {\em any} sequence of inputs, because it
is possible that for a given input $\tilde{x}$, the prover $\disP$ has "hardwired" the correct value $\tilde{y}=f(\tilde{x})$ and can compute it without investing 
any work. We therefore propose a definition that holds for inputs randomly chosen according to a given probability distribution $\cal D$, and we allow for
the possibility that the reward of a dishonest prover can be "negligibly" larger than the reward of the honest prover (for example if $\disP$ is lucky and such 
"hardwired" inputs are selected by $\cal D$).

\noindent
\begin{definition}[Sequential Rational Proof]
\label{def:SRP}
A rational proof $(P,V)$ for a function $f:$ $\bit^n$ $\to$ 
$\bit^n$ is $(\epsilon, K)$-{\sf sequentially composable} for an input distribution $\cal D$, if for every prover $\disP$, 
and every sequence of inputs 
$x,x_1,\ldots,x_k \in {\cal D}$ such that $C(x) \geq \sum_{i=1}^k 
\tilde{C}(x_i)$ and $k \leq K$ we have that $\sum_{i}\tilde{R}(x_i) - R \leq \epsilon$.
\end{definition}

% Some properties of Sequential Rational Proofs
\noindent
A few sufficient conditions for sequential composability follow.

\begin{lemma}
\label{lemma:cost-rew-ratios}
% Rew ratio ineq. implies SRP
Let $(P,V)$ be a rational proof.
If for every input $x$  it holds that $R(x)=R$ and  $C(x)=C$ for constants 
$R \mbox{ and } C$, and the 
following inequality holds for every 
$\disP\neq 
P$ and input $x\in {\cal D}$:
\[ \frac{\tilde{R}(x)}{R} \leq \frac{\tilde{C}(x)}{C} + \epsilon\]
then $(P,V)$ is $(KR\epsilon, K)$-sequentially composable for $\cal D$
\end{lemma}

\begin{corollary}
\label{cor:prob}
Let $(P,V)$ and $\rew$ be respectively an interactive proof and a reward 
function as in 
Definition \ref{def:RP}; if $\rew$ can only assume the values $0$ and $R$ for 
some constant $R$, let $\pDisR = \Pr[\rew((\disP,V)(x)) = R]$. If for $x \in {\cal D}$
$$  \pDisR \leq \frac{\tilde{C}(x)}{C} + \epsilon $$
then $(P,V)$ is $(KR\epsilon, K)$-sequentially composable for $\cal D$. 
% If rew can be only R or 0 then the sufficient condition is on the probability
\end{corollary}


\subsection{A weaker form of sequential composability}


The definition of sequential composability presented above aims at modeling a prover that works under \emph{a constrained budget}. In this case we want to ensure that acting as the honest prover would be the most profitable strategy.
In other scenarios, however, we may assume the prover having an \emph{unlimited budget}. For example, in the case of volunteering computation we each volunteering client's cost can be expressed by time/electricity with no "cap". Such a client would be looking for the most efficient way to use its resources for \emph{each delegated problem}, rather than for each allocation of the honest cost $C$.
To model this scenario we can consider a simpler definition of sequential composability where we require the honest prover to be maximizing not only reward but utility, defined as reward minus cost. 
Such a definition would be a weaker form of sequential composability as defined above: Definition \ref{def:SRP} would imply the notion sketched here. In the remainder of this work we would focus only on the stronger definition of sequential composability for provers with budget constraints.

%\section{On designing rational proofs as interactive proofs}
%\input{rps_as_ips.tex}


% Rational Proofs for NL 
% Are these better than the ones we have already?? (not round complexity-wise)
\section{Rational Proofs for Nondeterministic Space-Bounded Computations }
\label{sec:protocol}
We are now ready to present our protocol. The main idea it exploits is the concept of \emph{configuration graph} which we will present here only informally\footnote{The interested reader may consult an introductory textbook on computational complexity such as \cite{arora2009computational}.}. A configuration of a Turing Machine is a complete description of the current state of the computation: the state of $M$, the position of its heads, the non-blank values on its tapes.  Let $M$ be a (possibly nondeterministic) TM and $x$ a string.
The set of nodes in the configuration graph $G_{M,x}$  of $M$ when executed on input $x$ is the set of configurations of $M$ with $x$ on its $M$'s input tape. There is an arc from configuration $C_i$ to configuration $C_j$ if and only if any of the transition functions of $M$ produces $C_j$ when given in input the information in configuration $C_i$.

We can reduce the problem of whether $M$ accepts an input $x$ to a reachability problem in its configuration graph: in fact $M(x) = 1$ if and only if there exists a path from a starting configuration to an accepting one. If $M$ terminates in polynomial time then such a path must be of length at most polynomial.

The protocol presented below is a more general version of the one used in \cite{cg15} and very simple in structure. It consists of a "chasing game" between the verifier and the prover, where the prover "commits" at each step to an intermediate configuration. If the prover is cheating the configuration may or may not follow from the initial configuration or lead to the final accepting configuration. At each step and after $P$ communicates the intermediate configuration $C'$ the verifier then randomly chooses whether to continue invoking the protocol on the left or the right of $C'$. The protocol terminates when $V$ ends up on two previously declared adjacent configurations that he can check.  Intuitively, the protocol works since, if $x$ is not in the language, for any possible sequence of the prover's messages, there is at least one choice of random coins that allows $V$ to detect it; the space of such choices is polynomial in size.

We assume that $V$ has oracle access to the input $x$.
%\clearpage
\noindent What follows is a formal description of the protocol.
\begin{framed}
\begin{enumerate}
    \item $P$ sends to $V$:
    \begin{itemize}
    \item $C_{N}$, the final accepting configuration (the starting configuration, $C_1$, is known to the verifier);
    \item $N$, the number of steps between the two configurations. % Does it need to? sort of yes. What if the guy lies. It should be discussed possibly.
    \end{itemize}
    \item Then $V$ invokes the procedure $\PathCheck(N, C_{1}, C_{N})$.
\end{enumerate}
\end{framed}

\medskip
\noindent The procedure $\PathCheck(m, C_l, C_r)$ is defined for $1 \leq m \leq N$ as 
follows:
\begin{framed}
\begin{itemize}
    \item If $m > 1$, then:
    \begin{enumerate}
        \item $P$ sends intermediate configurations $C_{p}$ and $C_q$ (which may coincide) where $p = \lfloor \frac{l+m-1}{2} \rfloor$  and 
        $q = \lceil \frac{l+m-1}{2} \rceil$. % (m == r-l+1)
        \item If $p \neq q$, $V$ checks whether there is a transition leading from configuration $C_p$ to configuration $C_q$. If yes, $V$ accepts; otherwise $V$ halts and rejects.
	\item $V$ generates a random bit $b \in_R \bit$
        \item If  $b = 0$ then the protocol continues invoking $\PathCheck(\lfloor \frac{m}{2} \rfloor, C_l, C_p)$; If $b = 1$ the protocol continues invoking $\PathCheck(\lfloor \frac{m}{2} \rfloor, C_q, C_r)$
    \end{enumerate}
    \item If $m = 1$, then $V$ checks whether there is a transition leading from configuration $C_l$ to configuration $C_r$. If $l=1$, $V$ checks that $C_l$ is indeed the initial configuration $C_1$. If $r=N$, $V$ checks that $C_r$ is indeed the final configuration sent by $P$ at the beginning. If yes, $V$ accepts; otherwise $V$ rejects.
\end{itemize}
\end{framed}

\medskip

\begin{theorem}
\label{thm:main}
$\NTISP[\poly(n), S(n)] \subseteq \DRMA[O(\log n), O(S(n)\log n), O(S(n)\log n)]$
\end{theorem}
\begin{proof}
% Efficiency
Let us consider the efficiency of the protocol above.
It requires $O(\log n)$ rounds.
Since the computation is in $\NTISP[\poly(n), S(n)]$, the configurations $P$ sends to $V$ at each round have size $O(S(n) \log n)$.
The verifier only needs to read the configurations and, at the last round, check the existence of a transition leading from $C_l$ to $C_r$. $V$ can carry out the latter test in $O(S(n) \log n)$ time.

% Soundness
Let us now prove that this is a rational proof with noticeable reward gap.
%by showing the protocol satisfies the hypothesis of Lemma \ref{lemma:ip2rp}. 
Observe that the protool has perfect completeness. 
Let us now prove that the soundness is at most $1 - 2^{-\log N} = 1 - \frac{1}{O(\poly(n))}$.
We aim at proving that, if there is no path between the configurations $C_1$ and $C_N$ then $V$ rejects with probability at least $2^{-\log N}$.
Assume, for sake of simplicity, that $N = 2^k$ for some $k$. We will proceed by induction on $k$. If $k=1$, $P$ provides the only intermediate configuration $C'$ between $C_1$ and $C_N$. At this point $V$ flips a coin and the protocol will terminate after testing whether there exists a transition between $C_1$ and $C'$ or between $C'$ and $C_N$. Since we assume the input is not in the language, there exists at most one of such transitions and $V$ will detect this with probability $1/2$.

Now assume $k > 1$. At the first step of the protocol $P$ provides an intermediate configuration $C'$. Either there is no path between $C_1$ and $C'$ or there is no path between $C'$ and $C_N$. Say it is the former: the protocol will proceed on the left with probability $1/2$ and then $V$ will detect $P$ cheating with probability $2^{-k+1}$ by induction hypothesis, which concludes the proof.
\end{proof}

\medskip
\noindent
The theorem above implies directly the following results:

\begin{corollary}
$ \L \subseteq \NL \subseteq \DRMA[O(\log n), O(\log^2 n ), O(\log^2 n )]$
\end{corollary}


\begin{corollary}
$ \SC \subseteq  \NSC \subseteq \DRMA[O(\log n), O(\polylog(n)), O(\polylog(n))]$
\end{corollary}


\section{Sequential Composability of our protocol}

\subsection{Assumptions on Prover and cost model}

% Some introduction here on "making a transition" vs guessing

% NOTE: "Being function of x" means that the prover is able to see the input
A (non-adaptive) prover's strategy on input $x$ can be represented by a string $s$, function of $x$, of length $N-1$ \footnote{We can assume the number of steps is fixed by assuming the computation is carried out by an oblivious TM.}.
We assume that each of the symbols $s_i$ is such that $s_i \in \{ \tau, \gamma \}$. Informally $\tau$ represents a transition the prover computed and $\gamma$ a "guessing" operation.
The cost of a prover using strategy $s$ on input $x$ is:
$$ Cost(s) = |\{ i : s_i = \tau \}| $$

We now need to have a connection on how computing more or fewer transitions helps a prover gain the full reward. 
To do this, we will make assumptions on how the probability of providing the correct configuration is affected by the transitions.
Let $C_i$ be the correct $i$-th configuration for the computation, i.e.  the configuration as computed by the honest prover (represented by a string of only $\tau$-s). Given an arbitrary prover $\tilde{P}$ we define the configuration $\tilde{C}_i$ as the configuration provided by $\tilde{P}$ when asked by the prover.
\newcommand{\pCorrConf}{\tilde{q}}
\newcommand{\pGuessing}{\epsilon_g}
Let $\pCorrConf_i$ be defined as $\Pr[\tilde{C}_i = C_i]$.
We assume that:
\begin{itemize}
\item $\pCorrConf_0 = 1$; any prover will report the correct input configuration (it would be irrational to do otherwise).
\item For $1 \leq i \leq T$:
\[
    \pCorrConf_i = \left\{\begin{array}{lr}
        \pGuessing, & \text{if } s_i = \gamma \\
        \pCorrConf_{i-1}, & \text{if } s_i = \tau 
        \end{array}
  \]
\end{itemize}

The assumptions above may not hold for any distribution of inputs. Thus the following definition:
\begin{definition}
Let $\mathcal{D}$ be an input distribution and $\epsilon_g > 0$ a real number. We say the $\epsilon_g$-\emph{Hardness of State Guessing Assumption} holds for $\mathcal{D}$ if for every $s$ the probabilities $q_i$ of a prover reporting the right configuration satisfies the equality above.
\end{definition}

\subsection{Proof of Sequential Composability}


% XXX: Under which distribution??
\begin{theorem}
Let $\mathcal{D}$ be a distribution for which it holds the $\epsilon_g$-Hardness of State Guessing Assumption. Under the cost model above, the protocol of Section  \ref{sec:protocol} is a ($KR\epsilon$, K)-sequentially composable rational proof under $\mathcal{D}$ for any $K \in \mathbb{N}, R \in \mathbb{R}_{\leq 0}$ where $\epsilon = T\epsilon_g(1 - \frac{1}{T})$.
\end{theorem}
\begin{proof}
% XXX: Fix proof and theorem statement accounting  for: (1) distribution; (2) R [see Corollary 1 statement from cg15]

Let $\disP$ be a prover which choose to perform $t$ transitions.
Let $T$ be the cost for the honest prover.

Observe that $\pDisR$ is the probability that $V$ makes the final check on one of the transitions correctly computed by $\disP$. An upper bound on the probability is:
$$ \pDisR \leq \frac{t + (T-t-1)\epsilon_g}{T}$$

We can use the sufficient condition in Corollary $1$ from \cite{cg15}.
It suffices to show that 
$$\pDisR \leq \frac{t}{T} + \epsilon$$
To show this, we can use the observation above and show that:
$$ \frac{t + (T-t-1)\epsilon_g}{T}  \leq \frac{t}{T} + T\epsilon_g(1 - \frac{1}{T}) $$

Since the inequality above always holds, the proof is complete.
\end{proof}

% Includes: "crypto" extensions (observation on our protocol and on the PCP one), "impossibility" results for NP, composition thm, RPs for randomized computation.

\section{A composition theorem for rational proofs}

% Description of the composition theorem here

\section{Some general results on Rational Proofs}

\subsection{On Rational Arguments and cryptography}

\textbf{ TODO:Give definition of rational argument.}

\textbf{Lemma: }if something is a rational proof when V has oracle access to a certain string of length S then if CRHF exist there is a rational argument for the same language when V does not have that oracle access and the communication complexity is increased by polylog(k,n) (as well as verification complexity) % Remember the security parameter.

\textbf{Corollary: }if CRHF exist then P has a rational argument with 1 round (use lemma above on the PCP-like construction by Azar and Micali)

\textbf{Theorem: }if CRHF exist  then there is a rational argument for NP with sublinear [...] 
(The construction uses the idea for 3SAT (sample one clause and check it) and then combines it with the composition theorem)

\textbf{ Corollary: } if CRHF exist then there is a O(1) rational argument for NP with sublinear... [use the theorem above plus the theorem by Rosen et al. to tranform rational proofs in rational arguments. That thm assumes PIR which are implied by CRHF, I believe]

% NOTE: Important: write comparison with Kilian, i.e. ours is a _sublinear_ verification time delegation scheme for NP.

\subsection{Tighter lower bounds for efficient rational proofs for NP}

\textbf{Theorem: } If there exists a rational proof with noticeable reward gap for NP then $\NP \subseteq \bigcup_{k>0}\BPTIME[2^{O(log^k(n))} ]$ 

Proof: Use the technique from Rosen et al. (rational arguments) and show that the complexity of the PTM is the one claimed.

\textbf{NOTE: } Why is it that the results above about arguments (rather than proofs) for NP do not contradict the result above?

\section{Rational Proofs for Randomized Computations}

Goal: A simple approach to extending RPs to randomized computation. We will assume a Common Random String model and that V has Random Access to the string.

The theorems are:
Let $L \in BPSPACETIME(S(N), poly(n))$. If there are rational proofs for $DSPACETIME(S(n), poly(n))$ then there exists a rational proof for deciding L with the same efficiency parameters (up to a constant).

Implications of the theorem:
\begin{itemize}
\item we can do all$ BPSPACE[polylog, poly]$ with a polylog verifier thanks to the protocol we have in the rejected paper. 
\begin{itemize}
	\item (Notice that although $BPSPACETIME[polylog, poly] \subseteq NSPACETIME[polylog, poly]$ and for the latter class we already showed rational proofs (in the same paper). \textit{However} those rational proofs are one-sided (see comments from one of the reviewers), whereas this gives us two-sided proofs for BPSPACETIME!)
\end{itemize}
\item description We can also do very efficient (polylog everywhere or less) rational proofs for randomized circuits with bounded depth  (RNC).
\end{itemize}

Main idea of the protocol: 
\begin{itemize}
\item P sends V the result of the computation (0 or 1)
\item P computes poly(n) iterations of M(r;x) using the randomness from the CRS.
\item V selects one of these iterations at random, i.e. it selects a "chunk" r of randomness from the CRS -- Notice that V does not have to read it at all for now)
\item P communicates the result of the execution of M(r;x)
\item V pays P using the BSR (using as a "sample" the result of M(r; x) and as a "alleged distribution" whether P told x is in L or not) [Alon et al. do something like this in a piece in their first paper]
\end{itemize}

Proof sketch:
Assume that V has oracle access to M(r;x) (to be precise, to keep efficiency down, assume that V only has to specify the log(n)-sized index of the iteration to tell the oracle which randomness to use).
Under this assumption the one above is a Rational Proof for L. To finish proof use composition theorem of rational proofs to replace the oracle (notice that we are under the assumptions of the composition theorem because the randomness is public, it's considered part of the input and we have random access to it).


\subsection{ How to reduce the length of the CRS }

So far we assumed we had all the "randomness we needed" for all iterations in the CRS.
A PRG can stretch m ranndom bits to $m^d$ for a constant d. We believe PRGs to be computable in NC0 and we know they can be computable in NC1. We have very efficient rational proofs in both cases.
Assume a CRS $\sigma$ with length $m = O(n^\rho)$ with$ \rho < 1$ (notice that m has to be superpolylog anyway to support poly repetitions). Then we can let the prover use the PRG to generate the necessary randomness
from $\sigma$. The protocol above can then be changed by replacing a random access to the CRS to a rational proof to for the j-th bit of the output of the PRG. By the composition thm this should work.





\section{Conclusions and Open Problems}
We presented a rational proof for languages recognized by non-deterministic space-bounded computation. Our protocol is the first rational proof for a general class of language that does not use circuit representations. Our protocol is secure both in the standard stand-alone notion of rational proof introduced in 
\cite{am} and in the stronger composable version presented in \cite{cg15}. 

Our work leaves open a series of questions
\begin{itemize}
\item What is the relationship between scoring rule based protocols vs weak interactive proofs? Our work and the work in \cite{cg15} seem to indicate that the latter technique is more powerful (our work shows an example of a class of language which is not known to be recognizable using scoring rules, the work in \cite{cg15} shows that scoring rules seem inherently insecure in a composable setting). Is it possible to show, however, that a scoring-rule based protocol can be transformed into a weak interactive proof (without a substantial loss of efficiency) therefore showing that it is enough to focus on the latter?

\item Can we build efficient rational proofs for arbitrary poly-time computations, where the verifier runs in sub-linear, or even in linear, time? Even in the standalone model of \cite{am}?

\item Our proof of sequential composability considers only non-adaptive adversaries, and enforces this condition by the use of timing assumptions. Is it possible to construct protocols that are secure against adaptive adversaries? \textbf{XXX: Deal with next sentences before submitting.} Or is it possible to relax the timing assumption to something less stringent than what is required in our protocol?

\item It would be interesting to investigate the connection between the model of Rational Proofs and the work on Computational Decision Theory in  \cite{halpern2011don}. In particular it would be interesting to look at realistic cost models that could affect the choice of strategy by the prover particularly in the sequentially composable model. 
\end{itemize}




%\section*{Acknowledgement}
%We thank Jesper Buus Nielsen for the insightful discussions that inspired this work.
    
\bibliographystyle{plain}
\bibliography{references}

\end{document}
