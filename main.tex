
\documentclass{llncs}
\pagestyle{plain}
% XXX: Page numbering is not working with this either

\usepackage{times}
\usepackage{complexity}
\usepackage{url}
\usepackage{latexsym}
\usepackage[utf8]{inputenc}
\usepackage{framed}
\usepackage{natbib}
\usepackage{graphicx}
%\usepackage{amsthm}
\usepackage{amssymb}
\usepackage{amsmath}
\usepackage{verbatim}
\usepackage{enumitem}
\usepackage{verbatim}
%\usepackage{accents}
\usepackage{bm}
%\usepackage[normalem]{ulem}
%\setlength\titlebox{6.5cm}    % You can expand the title box if you
% really have to

\DeclareMathOperator*{\argmin}{arg\,min}
\DeclareMathOperator*{\argmax}{arg\,max}

\title{Sequentially Composable Rational Proofs for Space Bounded Computations}

\author{}

\institute{}
\date{}

\begin{document}
\maketitle
% Tau
\newcommand{\Tau}{\mathcal{T}}

% underbar
%\newcommand{\ubar}[1]{\underaccent{\bar}{#1}}
\newcommand{\ubar}[1]{\uline{#1}}

% Theorems
%\newtheorem{mydef}{Definition}
%\newtheorem{mylemma}{Lemma}
%\newtheorem{mytheorem}{Theorem}
%\newtheorem{mycorollary}{Corollary}

\newtheorem{myrem}{Remark}

% New complexity classes
%\newclass{\DTISP}{DTISP}
\newclass{\NTISP}{NTISP}
%\newclass{\SC}{SC}
\newclass{\NSC}{NSC}
%\newclass{\NC}{NC}
\newclass{\DRMA}{DRMA}
\newclass{\MRIP}{MRIP}

% Rational Proofs
%\newcommand{\A}{Arthur }
%\newcommand{\M}{Merlin }

\newcommand{\claimedy}{\tilde{y}}
\newcommand{\posreals}{\reals_{\geq 0}}

\newcommand{\cost}{c}
\newcommand{\costP}{\cost(P,x)}
\newcommand{\xVec}{\widetilde{X}}
\newcommand{\costDisPMany}{m\cdot\costDisP}
\newcommand{\costDisP}{\cost(\disP, x )}

\newcommand{\disP}{\widetilde{P}}
\newcommand{\disPSm}{\widetilde{P}^*}
\newcommand{\bfsP}{\widetilde{P}_{BFS}}

\newcommand{\prCh}{p_{cheat}}
\newcommand{\ratioCosts}{\frac{\cost_H(x)}{\disCLB}}

\newcommand{\function}[1]{\ensuremath{\mathsf{#1}}}

\newcommand{\Size}{\function{Size}}
\newcommand{\out}{\function{out}}
\newcommand{\rew}{\function{rew}}
\newcommand{\profit}{\function{profit}}
%\newcommand{\poly}{\function{poly}}
\newcommand{\negl}{\function{neg}}
%\newcommand{\reward}{\function{reward}}
%\newcommand{\log}{\function{log}}

\newcommand{\invPoly}{\frac{1}{\poly}}

\newcommand{\expRewProtDis}{\expectation[\rew((\disP,V)(x))]}
\newcommand{\expRewProtDisMany}{\expectation[\rew((\disP,V)(\xVec))]}
\newcommand{\expRewProtHon}{\expectation[\rew((P,V)(x))]}
\newcommand{\prOutProtDis}{\Pr[\out((\disP,V)(x)) \neq f(x)]}
\newcommand{\prOutProtDisMany}{\Pr[\out((\disP,V)(\xVec)) \neq f(\xVec)]}

% expected profit variants
\newcommand{\expProfitProtDis}{\expectation[\profit((\disP,V)(x))]}
\newcommand{\expProfitProtDisMany}{\expectation[\profit((\disP,V)(\xVec))]}
\newcommand{\expProfitProtHon}{\expectation[\profit((P,V)(x))]}

\newcommand{\pDisR}{\tilde{p}_x}

\newcommand{\rnd}{\rho}

\newcommand{\PathCheck}{\function{PathCheck}}

\newcommand{\circDFS}{\cal{C}_{DFS}}
\newcommand{\circBFS}{\overline{\cal{C}}_{BFS}}
\newcommand{\circNA}{\overline{\cal{C}}_{NA}}

\newcommand{\circEval}{\cal{C}_{\mbox{eval}}}
\newcommand{\circFFT}{\cal{C}_{\mbox{FFT}}}
\newcommand{\circPow}{\cal{C}_{\mbox{pow}}}

\newcommand{\tildeltaFFT}{\tildelta_{\mbox{FFT}}}

\newcommand{\probCheatFFT}{\tilde{p}_{\mbox{FFT}}}
\newcommand{\probCheatPow}{\tilde{p}_{\mbox{pow}}}

\newcommand{\tildelta}{\tilde{\delta}}
\newcommand{\investTildelta}{\tilde{C}_{\tildelta}}

\newcommand{\deltaBfsInv}{\delta_{BFS}(\tilde{C})}

\newcommand{\disR}{\tilde{R}}
\newcommand{\disRUB}{R'}
\newcommand{\disCLB}{\cost'}

\newcommand{\rewGapOneTime}{\Delta^{\rew}_{1}}
\newcommand{\profGapSeq}{\Delta^{\profit}_{\infty}}
\newcommand{\rewRatio}{\alpha^{\rew}}

% Announced distribution
\newcommand{\annd}{\hat d}
% Real distribution
\newcommand{\reald}{d}

%\newcommand{\expectation}{\mathbb{E}}
\newcommand{\expectation}{\mathop{\mathbb{E}}}

% Some shortcuts for math symbols
\newcommand{\binstrings}{\{0, 1\}^{*}}
\newcommand{\bits}{\{0, 1\}^{*}}
%\newcommand{\iff}{\Leftrightarrow}
\newcommand{\bit}{\{0, 1\}}
\newcommand{\funonstrings}{: \binstrings \to \binstrings}

\newcommand{\naturals}{\mathbb{N}}\label{key}
\newcommand{\reals}{\mathbb{R}}
\newcommand{\field}{\mathbb{F}}

% Circuit Family
\newcommand{\circfam}{\{C_n\}_{n=1}^{\infty}}
\newcommand{\langL}{\mathcal{L}}
% Interactive Proofs
\newcommand{\transc}{\mathcal{T}}

\makeatletter
\newcommand{\verbatimfont}[1]{\def\verbatim@font{#1}}%
\makeatother

\newclass{\BPTISP}{BPTISP}
\newclass{\BPNC}{BPNC}
\newclass{\osDRMA}{osDRMA}
\newclass{\BPRMA}{BPRMA}
%\newclass{\BPQP}{BPQP}
\newclass{\Cclass}{C}

\newcommand{\cb}[1]{\colorbox{BurntOrange}{#1}}
\newcommand{\CN}{\cb{\textbf{[CN]}}}
\newcommand{\XXX}{\cb{\textbf{[XXX]}}}

\newtheorem{definition}{Definition}
\newtheorem{lemma}{Lemma}
\newtheorem{claim}{Claim}
\newtheorem{theorem}{Theorem}
\newtheorem{corollary}{Corollary}



\newcommand{\DOM}{\function{DOM}}
\newcommand{\F}{\mathbb{F}}

\newcommand{\allhonest}{\function{all\_honest}}
\newcommand{\true}{\function{true}}
\newcommand{\false}{\function{false}}

\newcommand{\Enc}{\function{Enc}}
\newcommand{\pk}{\function{pk}}
\newcommand{\sk}{\function{sk}}

%% Shortcuts specific to Composition
\newcommand{\protOne}{\pi^{f_2}_{1}}
\newcommand{\protTwo}{\pi_{2}}
\newcommand{\rewGap}{\Delta}
\newcommand{\STEP}{STEP}

\newcommand{\disTransc}{\tilde{\Tau}}




%% Shortcuts specific to Composition
\newcommand{\protOne}{\pi^{L_2}_{1}}
\newcommand{\protTwo}{\pi_{2}}
\newcommand{\rewGap}{\Delta}
\newcommand{\STEP}{\function{STEP}}

\newcommand{\disTransc}{\tilde{\Tau}}


% Question: Come rendere l'abstract (e il paper) il piu' Eurocrypt-appetibile possibile?
% XXX: The abstract should be rewritten
\begin{abstract}
We present new protocols for the verification of {\em non-deterministic space bounded polytime computations} against a rational adversary.
For non-deterministic polytime computations in sublinear space our protocol requires only a verifier running in sublinear-time. Our protocols can be proven 
secure even when composed sequentially, a scenario where many previously proposed rational proofs fail. Our results are the first rational proofs not based on the circuit model of computation, and the first sequentially composable protocols for a well-defined language class. 
%Unlike previous rational protocols in literature, our protocol can be used for repeated delegations when the computation is costly for the prover. % XXX: %Citations to be included?
\end{abstract}


% -- BEGIN Old abstract --
\begin{comment}
\begin{abstract}
This paper presents new protocols for the verification of {\em non-deterministic space-bounded polytime computations} against a rational adversary. More specifically consider a language $L \in \NTISP(T(n), S(n))$, i.e. a language recognized by a non-deterministic Turing Machine $M_L$ which runs in time $T(n)$ and space $S(n)$. 
We construct a protocol where a rational prover can
convince the verifier that $x \in L$ with the following properties: 
\begin{itemize}
\item The verifier runs in time $O(S(n) \log n)$;
\item The protocol terminates in $O(\log n)$ rounds and communication complexity $O(S(n) \log n)$;
\item The prover simply runs $M_L(x)$ and stores all the intermediate configurations (i.e. requires space $O(S(n) T(n))$.
\end{itemize}
Note that for polytime computations in sublinear space this gives a protocol in which the verifier is sublinear-time. 
\end{abstract}
\end{comment}
% -- END Old abstract --

\section{Introduction}
The problem of efficiently checking the correctness of a computation performed by an untrusted party has been central in Cryptography and Complexity Theory for the last 30 years since the introduction of Interactive Proofs by Babai and Goldwasser, Micali and Rackoff \cite{babai,gmr}. 

The rise of the {\em cloud computing} paradigm has refocused our community's research effort towards finding solutions that are efficient and feasible in practice. How do we check the integrity of data that is stored remotely? How do we check computations performed on this remotely stored data? How can a client do this in the most efficient way possible?

Under the generic name of {\sf Verifiable Outsourced Computation} we have a very active research area in Cryptography and Network Security (see \cite{wb15} for a survey), to design protocols where it is impossible (under suitable cryptographic assumptions) for a provider to "cheat" in the above scenarios. While much progress has been done in this area, we are still far from solutions that can be deployed in practice. 

Part of the reason is that Cryptographers consider a very strong adversarial model that prevents {\sf any} adversary from cheating. A different approach is to restrict ourselves to {\em rational adversaries}, whose motivation is not just to disrupt the protocol or computation, but simply to maximize a well defined utility function (e.g. profit).

\medskip
\noindent
{\sc Our Contribution:}
This paper presents new protocols for the verification of {\em non-deterministic space-bounded polytime computations} against a rational adversary. More specifically consider a language $L \in \NTISP(T(n), S(n))$, i.e. recognized by a non-deterministic Turing Machine $M_L$ which runs in time $T(n)$ and space $S(n)$. 
We construct a protocol where a rational prover can
convince the verifier that $x \in L$ with the following properties: 
\begin{itemize}
\item The verifier runs in time $O(S(n) \log n)$
\item The protocol has $O(\log n)$ rounds and communication complexity $O(S(n) \log n)$
\item The prover simply runs $M_L(x)$ and stores all the intermediate configurations (i.e. requires space $O(S(n) T(n))$
\end{itemize}
Note that for polytime computations in sublinear space this gives a protocol in which the verifier is sublinear-time. 

\subsection{Relationship to Previous Work}
\label{sec:prior}

\noindent{{\sc Rational Proofs.}
The notion of proof that we use in our work is the concept of {\sf Rational Proofs} introduced by Azar and Micali in \cite{am} and refined in subsequent papers \cite{am1,ratargs}. 

In a Rational Proof, given a function $f$ and an input $x$, the server returns the value $y=f(x)$, and (possibly) some auxiliary information, to the client. The client will in turn 
pay the server for its work with a reward which is a function of the messages 
sent by the server and some randomness chosen by the client.  The crucial 
property is that this reward is maximized in expectation when the server 
returns the correct value $y$. Clearly a rational prover who is only interested 
in maximizing his reward, will always answer correctly. 

The most striking feature of Rational Proofs is their simplicity. For example in \cite{am}, Azar and Micali show {\sf single-message} Rational Proofs for any problem in $\#P$, where an (exponential-time) prover convinces a (poly-time) verifier of the number of satisfying assignment of a Boolean formula. 

For the case of "real-life" computations, Azar and Micali in \cite{am1} present $d$-round Rational Proofs for functions computed by (uniform) Boolean circuits of depth $d$, for $d=O(\log n)$. In this case the Verifier runs in logarithmic time.

Recent work \cite{ratsumchecks} shows how to obtain Rational Proofs with sublinear verifiers for languages in NC. Recalling that $\L \subseteq \NL \subseteq \NC_2$, one can use the protocol  in \cite{ratsumchecks} to verify a logspace polytime computation (deterministic or nondeterministic) in $O(\log^2 n )$ rounds and $O(\log^2 n )$ verification. For these computations our results compare
favorably to the one in \cite{ratsumchecks} in at least one aspect: our protocol requires $O(\log n )$ rounds and has the same verification complexity.

% Comparison between classes
This paper extends the line of work on rational proofs for "real-life" computation classes. We introduce the first rational proof for $\SC$ (also known as $\DTISP(T(n), S(n))$) with polylogarithmic verification and logarithmic rounds. To compare this with the results in \cite{ratsumchecks}, notice it is believed that $\NC \not = \SC$ and that the two classes are actually incomparable (see \cite{SCcompleteness} for a discussion).
Finally, our results provide the first efficient rational proof for the non-deterministic class $\NSC = \NTISP(\poly(n), \polylog(n) )$ . 

\medskip
\noindent{\sc Interactive Proofs.}
Obviously a "traditional" interactive proof (where security holds against any adversary, even a computationally unbounded one) would work in our model. In this case the most relevant result is 
the recent independent work in \cite{rrr16} that presents breakthrough protocols for the deterministic restriction of the class of language we consider. If $L$ is a language which is recognized by a deterministic Turing Machine $M_L$ which runs in time $T(n)$ and space $S(n)$, then their protocol has the following properties: 
\begin{itemize}
\item The verifier runs in 
$O(\poly(S(n)) + n \cdot\polylog(n))$ time;
\item The prover runs in polynomial time;
\item The protocol runs in {\em constant} rounds, with communication complexity $O({\sf poly}(S(n)n^{\delta})$ for a constant $\delta$.
\end{itemize}
Apart from round complexity (which is the impressive breakthrough of the result in \cite{rrr16}) our protocols fares better in all other categories. Note in particular that a sublinear space computation does not necessarily yield a sublinear-time verifier in 
\cite{rrr16}. On the other hand, we stress that our protocol only considers weaker rational adversaries. 

\medskip
\noindent{\sc Other Related Work.}
There is a large class of protocols for {\em arguments} of correctness (e.g. \cite{ggp10,ggpr13,krr14}) even in the rational model \cite{ratargs,ratsumchecks}. Recall that in an argument, security is achieved only against computationally bounded prover. In this case we can even achieve non-interactive solutions. But, in this paper, we are focused on what we can achieve without using computational assumptions.

% Halpern and Pass
Other works in theoretical computer science have studied the connections between cost of computation and utility in decision problems.
The work in \cite{halpern2011don} proposes a framework for \emph{computational decision problems}, where the Decision Maker's (DM) utility depends on the algorithm chosen for computing its strategy.
The Decision Maker runs the algorithm, assumed to be a Turing Machine, on the input to the computational decision problem.
The output of the algorithm determines the DM's strategy. 
Thus the choice of the DM reduces to the choice of a Turing Machine from a certain space. The DM will have beliefs on the running time (cost) of each Turing machine. The actual cost of running the chosen TM will affect the DM's reward.
Rational proofs with costly computation could be formalized in the language of \emph{computational decision problems} in \cite{halpern2011don}. There are similarities between the approach in this
work and that in \cite{halpern2011don}, as both take into account the cost of computation in a decision problem.

\begin{comment}
The cost models presented in our work could be extended by considering different type of cost functions, such as the ones in \cite{halpern2011don}. For example, we assume a cost function that is linear in the number of steps (transitions) carried out. A different approach could consider non-linear cost functions. What assumptions would be required to prove sequential composability when the cost of the computation increases, say, quadratically in the number of steps? 
And what are plausible cost functions in the real world?
\end{comment}




\subsection{Open Problems and Future Research}


The problem of interactive proofs for any polynomial-time computable function with an efficient verifier (e.g. quasi-linear) remains still open. It would be nice to be able to construct such proofs even simply in the rational model. 

\medskip
\noindent
In \cite{cg15} we presented a critique of the rational proof model in the case of "repeated executions with a budget". This model arises in the context of "volunteer computations" (\cite{seti,folding}) where many computational tasks are outsourced and provers compete in solving as many as possible to obtain rewards ). In this scenario assume that a prover has a certain budget $B$ of "computational effort": how can we guarantee that the rational strategy is to provide the correct answer in {\em all} the proof he provides? The notion of rational proof guarantees that if the prover engages in a single rational proof then it is in his best interest to provide the correct output. But in \cite{cg15} we showed that in the presence of many computations, it might be more profitable for the prover to use his budget $B$ to provide many incorrect answers than to provide a single correct answer. That's because incorrect (e.g. random) answers are "cheaper" to compute than the correct one and with the same budget $B$ the prover can provide many of them while the entire budget might be necessary to solve a single problem correctly. If the difference in reward between correct and incorrect answers is not high enough then many incorrect answers may be more profitable and a rational prover will choose that strategy. 

% XXX: Argue more about this?
The problem seems to be inherent in protocols based on scoring rules such as Brier's rule \cite{brier} (many, though not all, of the protocols in 
\cite{am,am1,ratargs,ratsumchecks} are based on such scoring rules). 

The protocol we present here can be thought of as an extension of the protocol in \cite{cg15}, which is based on a different approach. The Verifier does a limited check on the computation provided by the prover. This check is designed to be efficient (to keep the Verifier simple) and to catch a cheating prover with probability at least $1/{\sf poly}$. If the prover is not caught it gets a fixed reward $R$. This will guarantee that in the single-execution case the rational prover will always choose to provide correct answers and collect $R$ (since the cheating prover has expected reward at most $(1-1/{\sf poly})R$). 



% Interactive proofs

% XXX: From older version (TCC)
As we discussed in \cite{cg15} this type of protocols has a better chance to be secure (against rational adversaries) even in the "repeated" case, provided certain conditions happen. An interesting line of research which we plan to explore is to determine under what circumstances our new protocol for space bounded computation can be proven secure in the repeated case. 


\section{Preliminaries}
\section{Rational Proofs}

The following is the definition of Rational Proof from \cite{am}. As usual with $\negl(\cdot)$ we denote a {\em negligible} function, i.e. one that is asymptotically smaller than the inverse of any polynomial. Conversely a {\em noticeable} function is the inverse of a polynomial. 

% XXX: Do we need 
%In the following we will adopt a "concrete-security" version of the "asymptotic" 
%definitions and theorems in \cite{am1,ratargs}. We assume the reader is familiar with 
%the notion of interactive proofs \cite{gmr}. 

\noindent
\begin{definition}[Rational Proof]
\label{def:RP-delta}
\label{def:RP}
A function $f:$ $\bit^n$ $\to$ $\bit^*$ admits a rational proof if there exists an interactive proof $(P,V)$ and a randomized reward function
$\rew : \bits \to \posreals$ such that

\begin{enumerate}
\item \label{item:completeness} For any input $x \in 
\bit^n$, $\Pr[\out((P,V)(x)) = f(x)] \geq 1 - \negl(n).$

\item For every prover $\disP$, and for any input $x \in 
\bit^n$ there exists a $\delta_{\disP}(x) \geq 0$ such that 
$ \expRewProtDis + \delta_{\disP}(x) \leq \expRewProtHon. $
\end{enumerate}
The expectations and the probabilities are taken over the random coins of the prover and verifier.
\end{definition} 
We note that differently than \cite{am} we allow for non-perfect completeness: a negligible probability that even the correct prover will prove the wrong result. This will be necessary for our protocols for randomized computations. 


\medskip
\noindent
Let $\epsilon_{\disP} = \Pr[\out((P,V)(x)) \neq f(x)]$. 
Following \cite{ratargs} we define the {\sf reward gap} as 
\[ \Delta(x) = min_{P^* : \epsilon_{P^*}=1}[\delta_{P^*}(x)]  \]
i.e. the minimum reward gap over the provers that always report the incorrect value. 
It is easy to see that for arbitrary prover $\disP$ we have $\delta_{\disP}(x) \geq 
\epsilon_{\disP} \cdot \Delta(x)$. Therefore it suffices to prove that a protocol has 
a strictly positive reward gap $\Delta(x)$ for all $x$. 


% TODO: Ensure that $\Delta(x)$ is used consistently in all proofs.




\begin{definition}[\cite{am,am1,ratargs}]
The class $\DRMA[r, c, T]$ (Decisional Rational Merlin Arthur)
is the class of boolean functions $f : \bits \to \bit$ admitting a rational proof $\Pi = (P,V, \rew)$ s.t. on input $x$:
\begin{itemize}
    \item $\Pi$ terminates in $r(|x|)$ rounds;
    \item The communication complexity of $P$ is $c(|x|)$;
    \item The running time of $V$ is $T(|x|)$;
    \item The function $\rew$ is bounded by a polynomial;
    \item $\Pi$ has noticeable reward gap.
\end{itemize}
\end{definition}

\noindent
\begin{remark}
\label{rem:asy}
{\em The requirement that the reward gap must be noticeable was introduced in 
\cite{am1,ratargs} and is explained in Section~\ref{sec:proofs-seq-comp}.}
\end{remark}

\subsection{A Warmup Example}
\label{sec:example}

Consider the function $f:$ $\bit^n$ $\to$ $[0 \ldots n]$ which on input $x$ outputs the 
Hamming weight of $x$ (i.e. $\sum_{i} x_i$ where $x_i$ are the bits of $x$). 

In \cite{am1} the prover announces a number $\tilde{m}$ which he claims to be equal to $m=f(x)$. This can be interpreted as the prover announcing that if one chooses an input bit $x_i$ at random it will be equal to 1 with probability $\tilde{p}=\tilde{m}/n$. The verifier then chooses a random input bit $x_i$ and uses $\tilde{m},x_i$ to compute the reward via a scoring rule. Since the scoring rule is maximized by the announcement of the correct $m$, a rational prover will announce the correct value. The scoring rule used in \cite{am1} (and in all other rational proofs based on scoring rules) is 
Brier's rule where the reward is computed as $BSR(\tilde{p},x_i)$ where: 
\[BSR(\tilde{p},1) = 2\tilde{p}(2-\tilde{p}) \; \; \mbox{ and } \;\; 
BSR(\tilde{p},0) = 2(1-\tilde{p}^2) \]
Notice that $p=m/n$ is the actual probability to get 1 when selecting an input bit at random so the expected reqward of the prover is 
\begin{equation}
\label{eq:bsr}
p BSR(\tilde{p},1) + (1-p) BSR(\tilde{p},0) 
\end{equation}
which is easily seen to be maximized for $\tilde{p}=p$, i.e. $\tilde{m}=m$. 

in \cite{cg15} we propose an alternative protocol for $f$ (motivated by the issues we
discuss in Section~\ref{sec:proofs-seq-comp}). In our protocol we compute $f$ via an "addition circuit", organized as a complete binary tree with $n$ leaves which are the input, and where each internal node is a (fan-in 2) addition gate -- note that this circuit has depth $d=\log n$. The protocol has $d$ rounds: at the first round the prover announces $\tilde{m}$ (the claimed value of $f(x)$) and its two "children" $y_L,y_R$ in the output gate, i.e. the two input values of the last output gate $G$. The Verifier checks that 
$y_L +y_R=\tilde{m}$, and then asks the Prover to verify that $y_L$ or $y_R$ (chosen a random) is correct, by recursing on the above test. At the end the verifier has to check the last addition gate on two input bits: she performs this test on her own by reading just those two bits. If any of the tests fails, the verifier pays a reward of 0, otherwise she will pay $R$. The intuition is that a cheating prover will be caught with probability $2^{-d}$ which is exactly the reward gap (and for log-depth circuits like this one is noticeable). 

Note that the first protocol is a scoring-rule based one, while the second one is a weak-interactive proof. 




%\section{On designing rational proofs as interactive proofs}
%One of the striking differences between the rational protocols in \cite{} % XXX: Works by Micali, Rosen et al.
and classical interactive proofs is that the verifier carries out no check.
Roughly, the verifier would first exchange messages with the prover, then it would pass the transcript and a sample of a few bits from the input to a "black box function" that decides how much to pay the prover.
This black box satisfies the rationality property because it would yield the largest output, and thus the largest reward, only in the case of the prover's correct answer. The tool used in these protocols are \emph{scoring rules} \cite{brier}.  
A scoring rules take as argument a distribution $\mathcal{D}$ and a sample $\sigma$, allegedly retrieved according to $\mathcal{D}$. The return value of a scoring function is maximized on expectation only if $\mathcal{D}$ corresponds to the distribution from which $\sigma$ was sampled. In a rational proof, $\sigma$ would correspond to a few bits of the input and the distribution $\mathcal{D}$ would depend on the transcript. Intuitively, if a prover is cheating then the distribution $\mathcal{D}$ gets farther from the "honest" distribution and the prover's reward would decrease.

Scoring rules are not the only way to design a rational proof protocol. In fact, the following (trivial) result shows that we can build efficient rational proofs from efficient interactive proofs with poor soundness.

%\medskip
%\noindent

% Implicit theorem in our previous work (and in general, as an obvious fact):
\begin{lemma}
\label{lemma:ip2rp}
Every public coin protocol for function $f: \bits \to \bits$ with completeness 1, soundness at most $1-\frac{1}{\poly(n)}$ and verification time $T_V$  is a rational proof with noticeable reward gap for $f$ where $V$ runs in time $T_V$.
\end{lemma}
\begin{proof}
The verifier can run the interactive protocol and pay the Prover $R = \poly(|x|)$ if it accepts and 0 otherwise. 
\end{proof}

% Something else to be said here?

% TODO: Point out that it is necessary to have rationality as an assumption in order to obtain recent delegation schemes from these "poor" interactive proofs 
This fact suggests an approach to designing rational proofs starting from weak interactive proofs. Variants of this approach have already been used in literature: in \cite{am1} to rationally delegate problems in P (under the assumption of a PCP-like tape); in \cite{chen2016rational} to design a simple multi-prover protocol; in \cite{cg15}, where it is used for delegating functions computable by low-depth circuits. The protocol presented in this paper also follows this approach. These  protocols intuitively differ from the ones based on scoring rules for the presence of explicit checks. These checks have comparatively low probability of detecting a cheating prover, yet high enough to provide noticeable reward gaps to a reward-incentivized prover.

There are two main advantages to designing Rational proofs as traditional interactive proofs following the approach above. First, it guarantees a certain reward $R$ to the honest prover. In fact, if the honest prover is paid by scoring rules its reward will depend on the verifier's randomness. Although maximized in expectation, the reward for the honest prover can be anything between 0 and some maximum value when using scoring rules. Second, it may provide better chances of achieving sequential composability. One of the requirements of sequential composability is that a cheating prover carrying out a "very low-cost computation" will get a very low reward. However, as explained above, scoring rules can provide surprisingly high rewards even to "lazy" provers (see Section 4.1 in \cite{cg15}). It is unclear how this can be avoided with protocols based on scoring rules. On the other hand, some of the protocols in literature based on traditional interactive proofs can be proven sequentially composable in reasonable cost models, specifically the PCP-like protocol in \cite{am1} and the protocols in \cite{cg15} and in the current paper. It is unknown to the authors whether the interactive proof used as rational proof in \cite{chen2016rational} is also sequentially composable. Finally, the analysis of sequential composability for this type of protocols are comparatively easier to carry out as one can simply look at the connections between soundness, i.e. the probability for a dishonest prover of getting a high reward, the and cost to achieve it (see Corollary \ref{cor:prob}).
A disadvantage of the approach based on interactive proofs is that it requires more checks from the Verifier, thus possibly increasing the verification time and number of rounds.

% Open problems: all rational protocols as interactive protocols? Sequential composability only from those?
\subsubsection{Open problems}
It is not clear whether protocols based on scoring rules cannot be  sequentially composable. However, as mentioned above, sequential composability might be easier to prove in protocols based on traditional interactive proofs. For this reason it would be interesting to solve another open problem: whether, for any rational proof with certain complexity for a language, there exists a rational proof based on interactive proofs for the same language and with almost the same complexity (for example, requiring only an additional constant factor in the number of rounds). 



% Rational Proofs for NL 
% Are these better than the ones we have already?? (not round complexity-wise)
\section{Rational Proofs for Nondeterministic Space-Bounded Computations }
\label{sec:protocol}
We are now ready to present our protocol. The main idea it exploits is the concept of \emph{configuration graph} which we will present here only informally\footnote{The interested reader may consult an introductory textbook on computational complexity such as \cite{arora2009computational}.}. A configuration of a Turing Machine is a complete description of the current state of the computation: the state of $M$, the position of its heads, the non-blank values on its tapes.  Let $M$ be a (possibly nondeterministic) TM and $x$ a string.
The set of nodes in the configuration graph $G_{M,x}$  of $M$ when executed on input $x$ is the set of configurations of $M$ with $x$ on its $M$'s input tape. There is an arc from configuration $C_i$ to configuration $C_j$ if and only if any of the transition functions of $M$ produces $C_j$ when given in input the information in configuration $C_i$.

We can reduce the problem of whether $M$ accepts an input $x$ to a reachability problem in its configuration graph: in fact $M(x) = 1$ if and only if there exists a path from a starting configuration to an accepting one. If $M$ terminates in polynomial time then such a path must be of length at most polynomial.

The protocol presented below is a more general version of the one used in \cite{cg15} and very simple in structure. It consists of a "chasing game" between the verifier and the prover, where the prover "commits" at each step to an intermediate configuration. If the prover is cheating the configuration may or may not follow from the initial configuration or lead to the final accepting configuration. At each step and after $P$ communicates the intermediate configuration $C'$ the verifier then randomly chooses whether to continue invoking the protocol on the left or the right of $C'$. The protocol terminates when $V$ ends up on two previously declared adjacent configurations that he can check.  Intuitively, the protocol works since, if $x$ is not in the language, for any possible sequence of the prover's messages, there is at least one choice of random coins that allows $V$ to detect it; the space of such choices is polynomial in size.

We assume that $V$ has oracle access to the input $x$.
%\clearpage
\noindent What follows is a formal description of the protocol.
\begin{framed}
\begin{enumerate}
    \item $P$ sends to $V$:
    \begin{itemize}
    \item $C_{N}$, the final accepting configuration (the starting configuration, $C_1$, is known to the verifier);
    \item $N$, the number of steps between the two configurations. % Does it need to? sort of yes. What if the guy lies. It should be discussed possibly.
    \end{itemize}
    \item Then $V$ invokes the procedure $\PathCheck(N, C_{1}, C_{N})$.
\end{enumerate}
\end{framed}

\medskip
\noindent The procedure $\PathCheck(m, C_l, C_r)$ is defined for $1 \leq m \leq N$ as 
follows:
\begin{framed}
\begin{itemize}
    \item If $m > 1$, then:
    \begin{enumerate}
        \item $P$ sends intermediate configurations $C_{p}$ and $C_q$ (which may coincide) where $p = \lfloor \frac{l+m-1}{2} \rfloor$  and 
        $q = \lceil \frac{l+m-1}{2} \rceil$. % (m == r-l+1)
        \item $V$ generates a random bit $b \in_R \bit$
        \item If  $b = 0$ then the protocol continues invoking $\PathCheck(\lfloor \frac{m}{2} \rfloor, C_l, C_p)$; If $b = 1$ the protocol continues invoking $\PathCheck(\lfloor \frac{m}{2} \rfloor, C_q, C_r)$
    \end{enumerate}
    \item If $m = 1$, then $V$ checks whether there is a transition leading from configuration $C_l$ to configuration $C_r$. If yes, $V$ accepts; otherwise $V$ rejects.
\end{itemize}
\end{framed}

\medskip

\begin{theorem}
$\NTISP[\poly(n), S(n)] \subseteq \DRMA[O(\log n), O(S(n)\log n), O(S(n)\log n)]$
\end{theorem}
\begin{proof}
% Efficiency
Let us consider the efficiency of the protocol above.
It requires $O(\log n)$ rounds.
Since the computation is in $\NTISP[\poly(n), S(n)]$, the configurations $P$ sends to $V$ at each round have size $O(S(n) \log n)$.
The verifier only needs to read the configurations and, at the last round, check the existence of a transition leading from $C_l$ to $C_r$. $V$ can carry out the latter test in $O(S(n) \log n)$ time.

% Soundness
Let us now prove that this is a rational proof with noticeable reward gap by showing the protocol satisfies the hypothesis of Lemma \ref{lemma:ip2rp}. Observe that the completeness of the protocol above is 1. 
Let us now prove that the soundness is at most $1 - 2^{-\log N} = 1 - \frac{1}{O(\poly(n))}$.
We aim at proving that, if there is no path between the configurations $C_1$ and $C_N$ then $V$ rejects with probability at least $2^{-\log N}$.
Assume, for sake of simplicity, that $N = 2^k$ for some $k$. We will proceed by induction on $k$. If $k=1$, $P$ provides the only intermediate configuration $C'$ between $C_1$ and $C_N$. At this point $V$ flips a coin and the protocol will terminate after testing whether there exists a transition between $C_1$ and $C'$ or between $C'$ and $C_N$. Since we assume the input is not in the language, there exists at most one of such transitions and $V$ will detect this with probability $1/2$.

Now assume $k > 1$. At the first step of the protocol $P$ provides an intermediate configuration $C'$. Either there is no path between $C_1$ and $C'$ or there is no path between $C'$ and $C_N$. Say it is the former: the protocol will proceed on the left with probability $1/2$ and then $V$ will detect $P$ cheating with probability $2^{-k+1}$ by induction hypothesis, which concludes the proof.
\end{proof}

\medskip
\noindent
The theorem above implies directly the following results:

\begin{corollary}
$ \L \subseteq \NL \subseteq \DRMA[O(\log n), O(\log^2 n ), O(\log^2 n )]$
\end{corollary}


\begin{corollary}
$ \SC \subseteq  \NSC \subseteq \DRMA[O(\log n), O(\polylog(n)), O(\polylog(n))]$
\end{corollary}


\section{Sequential Composability of our protocol}
The definition of sequential rational proofs requires a relationship between the reward earned by the prover and the amount of "work" the prover invested to produce that result. In our protocol this means to demonstrate a link between the probability of success of a cheating prover and the amount of computation he invests.

The intuition is that to produce the correct result, the prover must run the computation and incur its full cost. Unfortunately 
this intuition is difficult, if not downright impossible, to formalize. Indeed for a specific input $x$ a "dishonest" prover $\disP$ could have the correct 
$y=f(x)$ value "hardwired" and could answer correctly without having to perform any computation at all. Similarly, for certain inputs $x,x'$ and a certain 
function $f$, a prover $\disP$ after computing $y=f(x)$ might be able to "recycle" some of the computation effort (by saving some state) and compute 
$y'=f(x')$ incurring a much smaller cost than computing it from scratch. 

A way to address this problem was suggested in \cite{b08} under the name of {\em Unique Inner State Assumption}: the idea is to assume a distribution 
$\cal D$ over the input space. When inputs $x$ are chosen according to $\cal D$, then we assume that computing $f$ requires cost $T$ from any party: 
this can be formalized by saying that if a party invests $t=\gamma T$ effort (for $\gamma \leq 1$), then it computes the correct value only with
probability negligibly close to $\gamma$ (since 
a party can always have a "mixed" strategy in which with probability $\gamma$ it runs the correct computation and with probability $1-\gamma$ does 
something else, like guessing at random). 

This assumption suffices in their scenario where the verifier checks the correctness of the prover's answer by running the computation herself. Our protocol has much less stringent requirements on the prover: the verifier just checks a single computation step in the entire process, albeit a step chosen at random among the entire sequence. We need to refine the above assumption in order to be able to prove our protocol sequentially composable. 

Informally our assumption states for every correct transition that the prover is able to produce he must pay "one" computation step. More formally for any Turing Machine 
$M$ we say that pair of configuration $C,C'$ is $M$-correct if $C'$ can be obtained from $C$ via a single computation step of $M$. 
\begin{definition}[Hardness of State Guessing Assumption]
\label{def:HSGA}
Let $M$ be a Turing Machine and let $L_M$ be the language recognized by $M$. We say that the {\em Hardness of State Guessing Assumption} holds for $M$, for distribution $\cal D$ and security parameter $\epsilon$ if for any machine $A$ running in time $t$ the probability that $A$ on input $x$ outputs more than $t$, $M$-correct pairs of configurations is at most $\epsilon$ (where the probability is taken over the choice of $x$ according to the distribution $\cal D$ and the internal coin tosses of $A$). 
\end{definition}


\subsection{Adaptive vs. Non-Adaptive Provers.}

Assumption~\ref{def:HSGA} guarantees that to come up with $t$ correct transitions, the prover must invest at least $t$ amount of work. We now move to the ultimate goal which is to link the amount of work invested by the prover, to his probability of success. 
As discussed in \cite{cg15} it is useful to distinguish between {\em adaptive and non-adaptive provers.}

Wehn running a rational proof on the computation of $M$ over an input $x$, an {\em adaptive} prover allocates its computational budget  {\em on the fly} during the execution of the rational proof. Conversely a {\em non-adaptive} prover $\disP$ uses his computational budget to compute as much as possible about $M(x)$ before starting the protocol with the verifier. Clearly an adaptive prover strategy is more powerful than a non-adaptive one (since the adaptive prover can direct its computation effort where it matters most, i.e. where the Verifier "checks" the computation).

As an example, it is not hard to see that in our protocol an adaptive prover can 
succesfully cheat without investing much computational effort at all. The prover will answer at random until the very last step when he will compute and answer with a correct transition. Even if we invoke Assumption~\ref{def:HSGA} a prover that invests only one computational step has a probability of success of  $1-\frac{1}{\poly(n)}$ (indeed the prover fails only if we end up checking against the initial configuration -- this is the attack that makes Theorem~\ref{thm:main} tight.). 


Is it possible to limit the Prover to a non-adaptive strategy? As pointed out in \cite{cg15} this could be achieved by imposing some "timing" constraints to the execution of the protocol: to prevent the prover from performing large computations while interacting with the Verifier, the latter could request that prover's responses be delivered "immediately", and if a delay happens then the Verifier will not pay the reward. Similar timing constraints have been used before in the cryptographic literature, e.g. see the notion of {\em timing assumptions} in the concurrent zero-knowedge protocols in \cite{dns}. Note that in order to require an "immediate" answer from the prover it is 
necessary that the latter stores all the intermediate configurations, which is why we require the prover to run in space $O(T(n)S(n))$ -- this condition is not needed for the protocol to be rational in the stand-alone case, since even the honest prover could just compute the correct transition on the fly. 

Therefore in the rest we assume that non-adaptive strategies are the only rational ones and proceed in analyzing our protocol 
under the assumption that the prover is adopting a non-adaptive strategy. 


\subsection{Proof of Sequential Composability}

Under Assumption~\ref{def:HSGA} the proof of sequential composability is almost 
immediate. 

\begin{theorem}
Let $L \in \NTISP[\poly(n), S(n)]$ and $M$ be a Turing Machine recognizing $L$. Assume that  Assumption~\ref{def:HSGA} holds for $M$, under input distribution 
$\cal D$ and parameter $\epsilon$. Then the protocol of Section  \ref{sec:protocol} is a $(KR\epsilon, K)$-sequentially composable rational proof under $\mathcal{D}$ for any $K \in \mathbb{N}, R \in \mathbb{R}_{\geq 0}$.  
\end{theorem}
\begin{proof}
Let $\disP$ be a prover with a running time of $t$ on input $x$.
Let $T$ be the total number of transitions required by $M$ on input $x$, i.e. the 
computational cost of the honest prover.

Observe that $\pDisR$ is the probability that $V$ makes the final check on one of the transitions correctly computed by $\disP$. 
Because of Assumption~\ref{def:HSGA} we know that the probability that $\disP$ can 
compute more than $t$ correct transitions is $\epsilon$, therefore an upperbound on 
$\pDisR$ is $\frac{t}{T}+\epsilon$ and the Theorem follows from Corollary~\ref{cor:prob}. 
\end{proof}

% Includes: "crypto" extensions (observation on our protocol and on the PCP one), "impossibility" results for NP, composition thm, RPs for randomized computation.

\section{A composition theorem for rational proofs}

% Description of the composition theorem here

\section{Some general results on Rational Proofs}

\subsection{On Rational Arguments and cryptography}

\textbf{ TODO:Give definition of rational argument.}

\textbf{Lemma: }if something is a rational proof when V has oracle access to a certain string of length S then if CRHF exist there is a rational argument for the same language when V does not have that oracle access and the communication complexity is increased by polylog(k,n) (as well as verification complexity) % Remember the security parameter.

\textbf{Corollary: }if CRHF exist then P has a rational argument with 1 round (use lemma above on the PCP-like construction by Azar and Micali)

\textbf{Theorem: }if CRHF exist  then there is a rational argument for NP with sublinear [...] 
(The construction uses the idea for 3SAT (sample one clause and check it) and then combines it with the composition theorem)

\textbf{ Corollary: } if CRHF exist then there is a O(1) rational argument for NP with sublinear... [use the theorem above plus the theorem by Rosen et al. to tranform rational proofs in rational arguments. That thm assumes PIR which are implied by CRHF, I believe]

% NOTE: Important: write comparison with Kilian, i.e. ours is a _sublinear_ verification time delegation scheme for NP.

\subsection{Tighter lower bounds for efficient rational proofs for NP}

\textbf{Theorem: } If there exists a rational proof with noticeable reward gap for NP then $\NP \subseteq \bigcup_{k>0}\BPTIME[2^{O(log^k(n))} ]$ 

Proof: Use the technique from Rosen et al. (rational arguments) and show that the complexity of the PTM is the one claimed.

\textbf{NOTE: } Why is it that the results above about arguments (rather than proofs) for NP do not contradict the result above?

\section{Rational Proofs for Randomized Computations}

Goal: A simple approach to extending RPs to randomized computation. We will assume a Common Random String model and that V has Random Access to the string.

The theorems are:
Let $L \in BPSPACETIME(S(N), poly(n))$. If there are rational proofs for $DSPACETIME(S(n), poly(n))$ then there exists a rational proof for deciding L with the same efficiency parameters (up to a constant).

Implications of the theorem:
\begin{itemize}
\item we can do all$ BPSPACE[polylog, poly]$ with a polylog verifier thanks to the protocol we have in the rejected paper. 
\begin{itemize}
	\item (Notice that although $BPSPACETIME[polylog, poly] \subseteq NSPACETIME[polylog, poly]$ and for the latter class we already showed rational proofs (in the same paper). \textit{However} those rational proofs are one-sided (see comments from one of the reviewers), whereas this gives us two-sided proofs for BPSPACETIME!)
\end{itemize}
\item description We can also do very efficient (polylog everywhere or less) rational proofs for randomized circuits with bounded depth  (RNC).
\end{itemize}

Main idea of the protocol: 
\begin{itemize}
\item P sends V the result of the computation (0 or 1)
\item P computes poly(n) iterations of M(r;x) using the randomness from the CRS.
\item V selects one of these iterations at random, i.e. it selects a "chunk" r of randomness from the CRS -- Notice that V does not have to read it at all for now)
\item P communicates the result of the execution of M(r;x)
\item V pays P using the BSR (using as a "sample" the result of M(r; x) and as a "alleged distribution" whether P told x is in L or not) [Alon et al. do something like this in a piece in their first paper]
\end{itemize}

Proof sketch:
Assume that V has oracle access to M(r;x) (to be precise, to keep efficiency down, assume that V only has to specify the log(n)-sized index of the iteration to tell the oracle which randomness to use).
Under this assumption the one above is a Rational Proof for L. To finish proof use composition theorem of rational proofs to replace the oracle (notice that we are under the assumptions of the composition theorem because the randomness is public, it's considered part of the input and we have random access to it).


\subsection{ How to reduce the length of the CRS }

So far we assumed we had all the "randomness we needed" for all iterations in the CRS.
A PRG can stretch m ranndom bits to $m^d$ for a constant d. We believe PRGs to be computable in NC0 and we know they can be computable in NC1. We have very efficient rational proofs in both cases.
Assume a CRS $\sigma$ with length $m = O(n^\rho)$ with$ \rho < 1$ (notice that m has to be superpolylog anyway to support poly repetitions). Then we can let the prover use the PRG to generate the necessary randomness
from $\sigma$. The protocol above can then be changed by replacing a random access to the CRS to a rational proof to for the j-th bit of the output of the PRG. By the composition thm this should work.





\section{Conclusions and Open Problems}
We presented a rational proof for languages recognized by non-deterministic space-bounded computation. Our protocol is the first rational proof for a general class of language that does not use circuit representations. Our protocol is secure both in the standard stand-alone notion of rational proof introduced in 
\cite{am} and in the stronger composable version presented in \cite{cg15}. 

Our work leaves open a series of questions
\begin{itemize}
\item What is the relationship between scoring rule based protocols vs weak interactive proofs? Our work and the work in \cite{cg15} seem to indicate that the latter technique is more powerful (our work shows an example of a class of language which is not known to be recognizable using scoring rules, the work in \cite{cg15} shows that scoring rules seem inherently insecure in a composable setting). Is it possible to show, however, that a scoring-rule based protocol can be transformed into a weak interactive proof (without a substantial loss of efficiency) therefore showing that it is enough to focus on the latter?

\item Can we build efficient rational proofs for arbitrary poly-time computations, where the verifier runs in sub-linear, or even in linear, time? Even in the standalone model of \cite{am}?

\item Our proof of sequential composability considers only non-adaptive adversaries, and enforces this condition by the use of timing assumptions. Is it possible to construct protocols that are secure against adaptive adversaries? \textbf{XXX: Deal with next sentences before submitting.} Or is it possible to relax the timing assumption to something less stringent than what is required in our protocol?

\item It would be interesting to investigate the connection between the model of Rational Proofs and the work on Computational Decision Theory in  \cite{halpern2011don}. In particular it would be interesting to look at realistic cost models that could affect the choice of strategy by the prover particularly in the sequentially composable model. 
\end{itemize}




%\section*{Acknowledgement}
%We thank Jesper Buus Nielsen for the insightful discussions that inspired this work.
    
\bibliographystyle{plain}
\bibliography{references}

\end{document}
